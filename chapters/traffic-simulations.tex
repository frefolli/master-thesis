\chapter{Traffic Simulations}
\label{chapter:traffic-simulations}

\section{SUMO}

\paragraph{\textbf{S}imulation of \textbf{U}rban \textbf{MO}bility}

SUMO \cite{krajzewicz2002sumo} is an open source multi-modal, microscopic traffic simulator developed by the Institute of Transportation Systems at the German Aerospace Center. It accept a network description file ($*.net.xml$), a demand and flows description file ($*.rou.xml$). It comes with several utilities which enable the user to construct or edit a network or a simulation, generate demand from OD matrices, verify correctness of description files, dump slices of OpenStreetMap and more. The simulator can run headless or within a graphical user interface (which can be seen in Figure \ref{fig:sumo-gui}), which let the user inspect details of network elements or vehicles at runtime and modify parameters at-fly. The user can also stop and resume the simulation as well as export its instant dump or change the simulation step.

\putimage{figures/sumo-gui.png}{A view of SUMO simulator GUI interface loaded with a simulation scenario. The picture is centered over a few intersections in which the phases of traffic lights are highlighted.}{fig:sumo-gui}{1.0}

\paragraph{\textbf{T}raffic \textbf{C}ommunication \textbf{I}nterface}

Since the simulator has been conceived as a standalone executable, there is not library to interact with and it can be controlled only with \textbf{TraCI}, a network application protocol defined as part of the SUMO project. Queries can be used to obtain data from the simulation, such as vehicles position, speed and route, lane density, traffic lights programs and state. A notable limitation is that there is no clear granularity for retrievable content. For example, it is possible to ask the \textit{mean waiting time of vehicles in a specific lane} but not the \textit{mean accumulated waiting time of vehicles of that lane}. The reason? Not a clue. Nonetheless, you may need to ask single pieces of information a query at a time, which is rather slow. For that reason, I recommend to cache and compute "manually" data whenever it is possible in order to avoid costly network connections. Ten queries per vehicle or per lane don't sound a lot until one discovers to have twenty-five thousand vehicles and a hundred lanes per simulation step just for a simple five intersection scenario. In further chapters I will explain which metrics I track, what they represent and why I need them. TraCI also allows to \textbf{set} some data, such as traffic light program and current state. This is important for implementing traffic light agents on top of SUMO. In further sections I will introduce the concept of multi agent simulation and how a traffic light can be seen as a proper agent.

\paragraph{\textbf{NET}work \textbf{EDIT}or}

SUMO has a scenario creation utility which enables the user to build and edit a network as well as demands and other additional data.
In Figure \ref{fig:netedit-network}, the graphical interface for shaping a network can be seen. It allows to create junctions and edges, editing intersection connections, adding traffic lights and TAZ and more.
In Figure \ref{fig:netedit-demand}, the graphical interface for adding demands can be seen. It allows to create routes, single vehicle trips, vehicle flows and more options to edit vehicle types and more.

\putimage{figures/netedit-network.png}{A view of SUMO Net-edit utility. This mode enables the user to modify the network of the simulation.}{fig:netedit-network}{1.0}
\putimage{figures/netedit-demand.png}{A view of SUMO Net-edit utility. This mode enables the user to modify the transit demand of the simulation.}{fig:netedit-demand}{1.0}

\paragraph{The network model}

% Edges
SUMO transport networks are mainly composed of edges and intersections. An \textbf{edge} is defined with a \textit{source} and a \textit{sink} intersection, it has a \textit{priority} and a \textit{type}, which declares the edge role within the network. \textit{Normal} edges represents real roads, while \textit{internal} edges are an abstraction used within intersections to configure where an incoming vehicle can go to. This last concept will be explained better when talking about intersections. There are also \textit{crossing} and \textit{walkingarea} edges, which are useful when dealing with simulations involving an active role of pedestrians. Each edge contains more \textit{lanes}.

% Lanes
Each lane defines its \textit{shape} as a sequence of coordinates (must be enabled via parameter $customShape$), its maximum \textit{speed} in meters per second, its logical length and its set of \textit{allowed} or \textit{disallowed} vehicles (as can be seen in Figure \ref{fig:netedit-permissions}). More will be said about vehicle types later on, but for now it suffices to say that this option is used to create bike lanes, tram ways and so on. Each lane also declares its ordinal \textit{index} with respect to its edge, which is useful when declaring connections between roads, junctions and lanes.

\putimage{figures/netedit-permissions.png}{A view of SUMO Net-edit utility. Setting permissions is possible with a dedicated panel on lanes and edges}{fig:netedit-permissions}{0.5}

% Intersections
Junctions are the nodes of the SUMO multi graph, and represent intersections (default) as well as other types of nodes. For example, a \textit{deadend} junction is used to represent an entry/exit point of the simulation. There are also \textit{internal} junctions which define a waiting position within the intersection. A junction has some \textit{incoming lanes} and some \textit{internal lanes}. The internal lanes, mentioned before, a the joints within the intersection which link an incoming lane with an internal edge and then an internal edge with an outcoming lane. Internal connections are highly customizable, as can be seen in Figure \ref{fig:netedit-internal-connections}, in which left turns are forbidden entirely (a situation encountered, for instance, in main urban roads such as viale Monza in Milan).

\putimage{figures/netedit-internal-connections.png}{A view of SUMO Net-edit utility. The (editable) internal connections of an intersection are highlighted. In blue the selected lane, in green the enabled turns, in yellow the lanes which SUMO suggests not to enable to avoid directional conflicts.}{fig:netedit-internal-connections}{0.5}

% Traffic Lights (and programs)
By default, intersections are regulated with the precedence rule, but traffic lights may be created in order to optimize the transit. A traffic light has one or more programs but has exacly one active program. A program is a sequence of states which will be hold for a fixed amount of time before changing to the next one (repeated cyclically). A state is declared as a string of characters ("\textbf{r}" for \textit{red}, "\textbf{y}" for \textit{yellow}, "\textbf{g}" for \textit{green} and "\textbf{G}" for \textit{priority green}), one for each connection. It is strongly recommended to mix \textit{green} (G) and \textit{priority green} (G) in order to avoid deadlocks and reduce unrealistic conflicts. For example, in Figure \ref{fig:netedit-tllogic} is shown an intersection in which straight and right directions are prioritized over left turns. Recalling what has been said about internal edges, a connection links an incoming lane and an outcoming lane in order to allow or disallow transits and turns. All the declared connections which are subject to a traffic light program are indexed with the ordinal parameter \textit{linkIndex}. This form of representation allows the user to reproduce any realistic scenario.

\putimage{figures/netedit-tllogic.png}{A view of SUMO Net-edit utility. The (editable) program of a traffic light is highlighted.}{fig:netedit-tllogic}{0.5}

% Roundabouts
Intersections can also be regulated with roundabout, which can be created via SUMO Net-edit utility as can be seen in Figure \ref{fig:netedit-roundabout}. These are not \textit{real} roundabouts, just a crooked representation of how a roundabout works. It sets some waiting positions inside a junction agglomerate in order to reproduce the cyclic behavior of vehicles in roundabout.
Interestingly, since those are created by blending more virtual junctions, one can have SUMO to add traffic lights to roundabout waiting positions, with questionable results.

\putimage{figures/netedit-roundabout.png}{A view of SUMO Net-edit utility. SUMO also supports simulation of roundabout, but are coerced into crooked intersections.}{fig:netedit-roundabout}{0.5}

\paragraph{The demand model}

Demands is expressed in terms of vehicles trips or vehicles flows. For more realistic scenarios, it is recommended to use the first option if an exact timeline of vehicles is available or if one is testing a specific situation that could happen inside inside a network slice. For example, GTFS data can be used to reproduce one-to-one the behavior of a public service network. An important notice is that the actual demands implementation is then randomized by SUMO via its simulation seed, so in order to be reproducible one should take note of the specific seed adopted for a given simulation run.

Both trips and flows declare a used vehicle type, which is the kind of entity generated into the simulation at runtime. As can be seen in Figure \ref{fig:netedit-permissions}, SUMO supports a large variety of vehicles types. Depending on the simulation aims, is it possible to include private vehicles such as \textit{cars} (there is also a distinction between internal combustion engines and electric vehicles), \textit{motorbikes}, \textit{trucks}, \textit{bicycles} but also \textit{pedestrians} and \textit{wheelchairs}. The choice for public service and non-road vehicles is wider: emergency vehicles like \textit{ambulances}, road public transit like \textit{taxis}, \textit{urban} and \textit{suburban buses}, railway vehicles such as \textit{trams}, \textit{light rail}, \textit{heavy rail} and \textit{subway} (with the distinction between electric/non-electric convoys), airborne vehicles such as \textit{cable cars}, \textit{airplanes} and \textit{drones}, special duty vehicles such as government/police/army vehicles and finally \textit{ships}.

Is it also possible to create a custom vehicle type with Net-edit, by declaring its \textit{vClass} (the parent vehicle type, e.g. tram), its physical properties like length, max speed, acceleration, deceleration and the gap it wants to maintain when queued to other vehicles.

Vehicles generated by trips and flows follow a route which can be statically allocated or change dynamically during the simulation. The SUMO simulator will check if those declared routes are continuous and valid before even starting the simulation: when using utilities other than Net-edit or other SUMO related software, the user must ensure that routes were generated correctly. There is also a specialized SUMO tool which checks routes correctness.

Inside the demands description file, a static route must be declared as a continuous sequence of waypoints (the "via" attribute, which can contain only edges). The Net-edit utility is very helpful because it lets the user to set a discontinuous sequence of waypoints and will use a routing algorithm in order to complete the selected route. For example, in Figure \ref{fig:netedit-trip-between-edges} I selected an edge in the bottom left corner and an edge in the top right corner. Enforcing the connection links defined in intersections and the lane vehicle type permissions, it has found an optimal \textit{valid} route between those waypoints. Net-edit also warns the user about unreachable edges for the current route, as can be seen in Figure \ref{fig:netedit-unreachable-edges}, where those edges are highlighted in pink.

\putimage{figures/netedit-trip-between-edges}{A route from edge-to-edge, completed automatically by Net-edit.}{fig:netedit-trip-between-edges}{0.5}
\putimage{figures/netedit-unreachable-edges.png}{Net-edit highlights in pink the unreachable edges for the current route.}{fig:netedit-unreachable-edges}{0.5}

Dynamic routes can instead be declared simply by stating an origin and a destination waypoint. One can create routes this way with edges, junctions and TAZs. At runtime, SUMO will generate a concrete route for those vehicles, which can be randomized and changed even during the trip execution if specified. It is important to notice that routes can also be defined independently from trips and flows: when creating such trips and flows, they can be simply attached to already declared routes.

I previously talked about junctions and edges in SUMO networks, but what are TAZs? Recalling what has been said in the last chapter about \textit{OD matrices}, the transport network can be divided into (maybe intersecting) zones containing junctions and edges. If a route is defined over TAZs as waypoints, the resulting vehicles will appear on edges covered/crossed by the origin TAZ and will be routed towards an edge covered/crossed by the destination TAZ. The same happens for intermediate TAZ waypoints. This allows to recreate realistic demands described by OD matrices. For example, in Figure \ref{fig:netedit-trip-between-tazs} I defined a route from a TAZ in the bottom left corner to a TAZ in the top right corner.

\putimage{figures/netedit-trip-between-tazs.png}{Trips can be created between TAZs.}{fig:netedit-trip-between-tazs}{0.5}

Once that vehicle types and routes are set, there are other options which can be used to customize the journeys. One can define the departure/arrival speed, lane, position. For example, the departure lane can be either set to \textit{free} (depart from a random available lane) or from a specified lane. This can create custom scenarios where vehicles can appear or disappear only at specified conditions. There also graphical options such as the vehicle color and logical options which can influence the simulation, like the number of vehicle occupants/passengers.

\paragraph{Flexibility}

In SUMO every simulation detail is highly customizable.
As an example, the user can create an ad-hoc \textit{vehicle type} with custom \textit{acceleration}, \textit{emergency deceleration}, \textit{startup delay}, \textit{maximum speed}, \textit{space gap} kept when following another vehicle, \textit{shape} and \textit{color} (useful for graphic simulations), \textit{dimensions} (length, width, height) and behaviour models such as the \textit{lane-changing} and the \textit{car-following}.

But customizability isn't limited to vehicles: also other simulation parameters can be changed at will. For example, depending on the degree of realism required, the user can decide the granularity of events such as the \textit{number of seconds} required to complete a \textit{lane change}, the \textit{delay of entrance} of vehicles provided by the demands configuration, the length in second of a \textit{simulation step} and much more.

There are also options which enable convenient behaviors. If the user wants to use junctions as TAZs (create demand flows between junctions), a specific option tells SUMO to insert a virtual TAZ for each intersection. Another useful option allows vehicles to ignore collisions when a deadlock happen in a intersection (which otherwise would result in a forever freezing state). The user can also set the number of seconds a vehicle may wait in a intersection until a deadlock is detected.

\section{Multi-agent simulations}

\paragraph{M.A.S.}

A Multi Agent System is composed of agents laying in an environment, sensing its state through observation and interacting with it through actions. Agents can be stateless or stateful, intelligent or statically programmed, and have different degree of autonomy. Agents can also influence each other with mediated or direct interactions. With SUMO, the simulation involves vehicles agents (such as cars, buses, pedestrians) and infrastructure agents (such as traffic lights).

\paragraph{Vehicles}

Vehicles observe the road, its occupancy and infrastructure signals in order to drive through an allocated route or in general go from a point A to a point B. They can make decisions whether to accelerate, decelerate, steer or change lane and they can base their actions on preference, knowledge or convenience. In SUMO, vehicles are usually assigned to a previously defined route, but can also made (through options) free to use an arbitrary lane or to re-route in case of \gls{gl:traffic} and persistent delays.

\paragraph{Traffic Lights}

Traffic lights have a state, which is the signal phase, and their only action is changing to another phase or maintaining the current one. Each phase usually decides which incoming lanes vehicles are allowed to enter the intersection and where they can exit. These agents can based their decision using occupancy and waiting time data from incoming and outcoming lanes. In SUMO, a traffic light entity manages a whole intersection.

\section{A control plane of SUMO agents}

In SUMO the agents have a fixed behaviour, but in multi agent system is common to have agents which are able to learn over time. Moreover, as SUMO supports only fixed cycle traffic lights, it can be useful to be able to customize their mode of operation in order to experiment different \gls{gl:traffic} management strategies in advance, thus avoiding to implement ruinous or useless schemas. Programmable control planes for SUMO over TraCI have been implemented for all these reasons.

A notable example of such control plane is \textbf{SUMO-RL} \cite{sumorl}, developed by Lucas Alegre. This is Python library designed to easily implement \textit{reinforcement learning} agents for managing SUMO traffic lights. It should be noted that the same interface can be used to implement deterministic and ''dumb`` agents.

A fork of this library \footnote{available on GitHub at \href{https://github.com/frefolli/sumo-rl}{https://github.com/frefolli/sumo-rl}} has been created in order to clean the architecture and implement more models useful for the present research experiments. Furthermore, I've implemented some utilities in order to convert data coming from other sources (such as CityFlow \cite{10.1145/3308558.3314139}, another free and open source, but abandoned, traffic simulator used as alternative to SUMO in literature), generate \gls{gl:traffic} data, analyze road networks, execute in batch a simulation experiment and track down multiple evaluation metrics. The contents and algorithms of this library will be explained in Chapter \ref{chapter:sumo-rl}.
